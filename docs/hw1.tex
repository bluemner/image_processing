\documentclass[12pt]{article}
\usepackage[]{algorithm2e}
\usepackage{ amssymb }
\usepackage{amsmath}
\usepackage[hyphens]{url}
\usepackage{listings}
\usepackage{xcolor}

\definecolor{listinggray}{gray}{0.9}
\definecolor{lbcolor}{rgb}{0.9,0.9,0.9}
\lstset{
    backgroundcolor=\color{lbcolor},
    tabsize=4,
    language=C++,
    captionpos=b,
    tabsize=3,
    frame=lines,
    numbers=left,
    numberstyle=\tiny,
    numbersep=5pt,
    breaklines=true,
    showstringspaces=false,
    basicstyle=\footnotesize,
    %  identifierstyle=\color{magenta},
    keywordstyle=\color[rgb]{0,0,1},
    commentstyle=\color{Darkgreen},
    stringstyle=\color{red}
}

\begin{document}
\title{Homework 1 CS-712}
\author{Brandon Bluemner}
\date{2017}
\maketitle
% //Start

    %% ____________________________________________________________________________
    %% Problem 1
    %% ____________________________________________________________________________
    \section{Problem 1}
      
        \begin{table}
        \centering
        \caption{S1 $   || \quad \quad   $ S2 $ \quad \ \ \ \ |||  $ S1 $\cup$ S2 $ \quad \quad $}


         \begin{tabular}{ | l | l | l | l | | l | l | l | l | | | l | l | l | l | }
            
            \hline
            0 & 0 & 0 & 0 &   0 & 0 & 1 & 1 &                        0 & 0 & \underline{\textbf{1}} & \underline{\textbf{1}}\\\hline
            0 & 0 & 1 & 0 &   0 & 1 & 0 & 0 &                        0 & \underline{\textbf{1}} & \underline{\textbf{1}} & 0\\\hline
            0 & 0 & 1 & 0 &   \underline{1} & 1 & 0 & 0 &   \underline{\textbf{1}} & \underline{\textbf{1}} & \underline{\textbf{1}} & 0\\\hline
            0 & 1 & 1 & \underline{1} &   0 & 0 & 0 & 0 &   0 & \underline{\textbf{1}} & \underline{\textbf{1}} & \underline{\textbf{1}}\\\hline
            \end{tabular}
            \label{tab:table1}
        \end{table}

        \subsection{a : 4 - adjacent}
        As show in the  \textbf{ Table \ref{tab:table1}} for 4 - adjacent the answer is no, 
        this is because $\forall i \in [0-3],   S1[i][3] \neq S2[i][0]$. 
        So the column 3 in S1 and the column 0 in s2 have no over laps with 
        respect to $V=\{1\}$

        \subsection{b : 8 - adjacent}
        Ths is an adjacent connection  show by the two underscored numbers
        in S1 and S2 ( $S1[3][3] = S2[0][3]$ on the diagonal )

        \subsection{Union Question}
        $ S1 \cup S2 $ (as shown in the far right of Table \ref{tab:table1}) is both
        4-adjacent, denoted by bold numbers and 8-adjacent denoted by underscored numbers.
    \newpage
    %% ____________________________________________________________________________
    %% Problem 2
    %% ____________________________________________________________________________
    \section{Problem 2}
    The images shown below are quite different, but their histograms are the same. 
    Suppose that each image is blurred with a 3*3 averaging mask. Would the 
    histograms of the blurred images still be equal? Please explain.
    
    No, The left side blue effect will have pixes such as the following
    \\
    \begin{tabular}{ | l | l | l | }
        \hline
        0 & 0 & 1\\\hline
        0 & 0 & 1\\\hline
        0 & 0 & 1\\\hline
        \end{tabular} 
     \begin{tabular}{ | l | l | l | }
            \hline
            0 & 1 & 1\\\hline
            0 & 1 & 1\\\hline
            0 & 1 & 1\\\hline
            \end{tabular}
    \\
    how ever the right had side will get a blur filer of 
    \\
    \begin{tabular}{ | l | l | l | }
        \hline
        1 & 0 & 0\\\hline
        0 & 1 & 1\\\hline
        0 & 1 & 1\\\hline
        \end{tabular}
    which will yield a intensity that the left image won't have. \\
    $\therefore$ the images will be different.
    %% ____________________________________________________________________________
    %% Problem 3
    %% ____________________________________________________________________________
    \section{Problem 3}
            Consider the image segment shown 
            \\
            \begin{tabular}{ l l l l l l  }
                & 3 & 1 & 2 & 1 & (q) \\
                & 2 & 2 & 3 & 2 & \\
                & 1 & 2 & 1 & 1 & \\
                (p) & 1 & 0 & 1 & 2 & \\
            \end{tabular}
            %% Problem 3 a
            \subsection{a : Let ${V=\{0,1\}}$}
                
                \begin{tabular}{ l l l l l l  }
                    & x & 1 & x & 1 & (q) \\
                    &  \underline{\textbf{x}} &  \underline{\textbf{x}} &  \underline{\textbf{x}} &  \underline{\textbf{x}} & \\
                    & 1 & x & 1 & 1 & \\
                    (p) & 1 & 0 & 1 & x & \\
                \end{tabular}
            \\
            \\

             There is no way for 4 -adjacent or 8 - adjacent to get from
             $p \to q$, as $row[2]$ has no members of the set as shown above in 
             underscored \underline{x}

             %% Problem 3 b
             \subsection{b : Let ${V=\{1,2\}}$}
            %% \underline{\textbf{x}}
             \begin{tabular}{ l l l l l l  }
                & x & \underline{\textbf{1}} & \underline{\textbf{2}} & \underline{\textbf{1}} & (q) \\
                & \underline{\textbf{2}} & \underline{\textbf{2}} & x & \underline{\textbf{2}} & \\
                & \underline{\textbf{1}} & \underline{\textbf{2}} & \underline{\textbf{1}} & \underline{\textbf{1}} & \\
                (p) & \underline{\textbf{1}} & x & 1 & 2 & \\
             \end{tabular}
             \\
             \\
             
             A path exists for both \textbf{4-adjacent} and \textbf{8-adjacent} highlighted above.
    %% ____________________________________________________________________________
    %% Problem 4
    %% ____________________________________________________________________________
    \section{Problem 4}
    Perform histogram equalization on the following histogram, where r is the intensity level 
    and n is the number of pixels with the corresponding intensity level. You need to find out
     the histogram after the equalization.
        \begin{tabular}{ | l | l | l | }
            \hline
            r & n & ${n} \div (M \times N)$\\\hline
            0 & 400 & 0.09765625\\\hline
            1 & 700 & 0.170898438\\\hline
            2 & 800 & 0.1953125\\\hline
            3 & 900 & 0.219726563\\\hline
            4 & 500 & 0.122070313\\\hline
            5 & 400 & 0.09765625\\\hline
            6 & 196 & 0.047851563\\\hline
            7 & 200 & 0.048828125\\\hline
            \hline        
            $\sum$  & 4096 & M = 64\\\hline
            $\sqrt{\sum}$ :: & 64  & N = 64\\\hline
        \end{tabular}
    \newpage
    %% ____________________________________________________________________________
    %% Problem 5
    %% ____________________________________________________________________________    
    \section{Problem 5}
    Give a 3*3 mask for performing unsharp masking in a single pass through an image.
    Assume that the average image is obtained using the filter below.
    \\
    \\
    $\frac{1}{9} \times$  \begin{tabular}{ | l | l | l | }
        \hline
        1 & 1 & 1\\\hline
        1 & 1 & 1\\\hline
        1 & 1 & 1\\\hline
    \end{tabular}
    \\
    \\
    $f(x,y)=$ original image, \ \ $\bar{f}(x,y)$= blur image, 
    $g_{mask}=f(x,y)-\bar{f}(x,y)$\\
    $g(x,y)= f(x,y)+g_{mask} \Rightarrow g(x,y)= f(x,y)+f(x,y)-\bar{f}(x,y)\\
    \therefore 2f(x,y)-\bar{f}(x,y) $ \\
    
    %% ____________________________________________________________________________
    %% Problem 6
    %% ____________________________________________________________________________
    \section{Problem 6}
    Textbook Problem 3.28
    Show that subtracting the Laplacian from an image is proportional to unsharp masking. 
    Use the definition for the Laplacian given below:
    \\
    \\
    $\bigtriangledown^{2}f(x,y)= f(x+1, y) + f(x-1,y) + f(x,y+1)+f(x,y-1)-4f(x,y)$ \\
    \\
    $f(x,y)-\bigtriangledown^{2}f(x,y)=$\\
    $=f(x,y)-[f(x+1, y) + f(x-1,y) + f(x,y+1)+f(x,y-1)-4f(x,y)]$\\
    $=6f(x,y)-[f(x+1, y) + f(x-1,y) + f(x,y+1)+f(x,y-1)+1f(x,y)]$\\
    $=5 \times [ \frac{6}{5} \times f(x,y)-$\\
    $ \frac{1}{5} \times [f(x+1, y) + f(x-1,y) + f(x,y+1)+ f(x,y-1)+1f(x,y)]]$\\
    \\
    $=5 \times [\frac{6}{5}f(x,y)-\bar{f}(x,y)]$ \\
    $\therefore  f(x,y) - \bigtriangledown^{2} f(x,y) \propto f(x,y)-\bar{f}(x,y) $
    %% ____________________________________________________________________________
    %% Problem 7
    %% ____________________________________________________________________________
    \section{Problem 7}
    Textbook Problem 4.12
    Consider a checkerboard image in which each square is 1mm * 1 mm. Assuming that the 
    image extends infinitely in both coordinate directions, what is the minimum sampling
     rate (in samples/mm) required to avoid aliasing?
    (Hint; consider the image in the form of . . . 101010101 . . . .The period
     of this signal is 2mm.)
     \\
     \\
     $P= 2mm; frequency = \frac{1}{period}$\\
     $max_{frequency} = \frac{1}{p} \Rightarrow max_{frequency} = \frac{1}{2}= 0.5 \frac{cycles}{mm} $  \\
     to avoid the aliasing $2\times max_{frequency} = 2(0.5) = 1$
     \\
     \\
     $\therefore$ each sample would need to exceed $\frac{1 sample}{mm}$

     \section{Problem 8}
     Prove that the 1DFT satisfies
     $$F(0)= \sum_{x=0}^{M-1}f_{x}$$

    $$F(u)= \sum_{x=0}^{M-1}f_{x} \times e^{\frac{-j2\pi ux}{M} }$$\\
    $e^{ \frac{-j2\pi ux}{M} }$ where $u=0 \Rightarrow  e^{0}=1$
    $$F(0)= \sum_{x=0}^{M-1}f_{x} \times 1 = \sum_{x=0}^{M-1}f_{x \ \blacksquare}  $$
\end{document}

